\begin{markdown}

## Performance Portablility ##

OpenCL \cite{opencl} was introduced in 2009 as an effort to make
programs portable between different hetrogeneous platforms. OpenCL is
a uniform interface to these platforms but the performance not proven
be portable \cite{komatsu2010evaluating}. Paper \cite{pocl} presents a
performance portable kernel compiler and a reference
implementation. The compiler is build with two layers, a host layer
that identifies parallel regions and a device layer that maps the
regions to the device hardware for execution. The compiler has proven
to be comparable in performance with the currently best
implementations on Intel Core i7, ARM Cortex A9 andSTI CellBE.
As of the time of writing no support for commercial GPUs exists.

### Parallel regions ###

The kernel compiler is implemented as a series of LLVM compiler
passes. They extract regions in the code that can be parallelized
without taking the underlying hardware in consideration. The compiler
then leaves it up to the device layer to map the parallel regions to
the parallel components of the hardware. 

\end{markdown}
